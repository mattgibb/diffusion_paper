%!TEX root = ../diffusion_paper.tex
\section{Introduction} % (fold)
\label{sec:introduction}
  \IEEEPARstart{H}{istological} sections are used in a broad range of biomedical applications, from basic biology to clinical use. In cardiac research, while other imaging modalities such as contrast-enhanced MRI \cite{Gilbert2012} or confocal microscopy \cite{Hooks2002,Rutherford2012} have been used to provide information about the microstructure of the heart, they are limited either in sample size or in their capability for cell type identification. Histology is the only modality able to comprehensively characterize cardiac microstructure, including the determination of individual cell types in the myocardium. 
  Histological images can be acquired on the un-cut surface of an embedded histological sample after sequential removal of thin slices (block-face images), in which case the fixed position of the sample with respect to the camera provides a natural inter-slice alignment \cite{Sands2005,Sands2006,Rutherford2012}. However, this restricts the possibilities for slice post-processing, limiting the differentiation of cardiac cells. Images of the individual 2D sections offer unparalleled level of detail \cite{Burton2006,Plank2009} but do not form a consistent 3D dataset, a major limitation when analyzing the complex 3D structure of the myocardium. Here we present a set of methods for accurate reconstuction of 3D cardiac anatomy from a combination of block-face and individual 3D sections.
  
  
  Previous approaches have used sequential registration between adjacent slices \cite{Chakravarty2006,Schmitt2006,Cifor2009,Cifor2011}. Adjacent whole-organ histological images are, in the main, similar in shape and colour profile and are obtained at the same spatial resolution. These characteristics make for highly accurate and reliable coregistration. However, volumes obtained in this way suffer from an artificial straightening of structures in the cross-slice orientation, known as the `z-shift effect' \cite{Yushkevich2006} or the `banana problem' (using as an example a banana whose sections are realigned, removing its curvature) \cite{Malandain2004,Lyon2012}. This effect is worsened with deformable registration, in the limit leading to all slices being identically shaped.
  
  Registering slices to a reference volume such as an MRI scan \cite{Alic2011,Osechinskiy2011,Kimm2012} circumvents the banana problem. Burton et al. \cite{Burton2006} combined MRI rabbit cardiac images of 26.4 x 26.4 x 24.4$\mu$m voxel size with 1.1 x 1.1 x 10$\mu$m histology stacks of the same hearts, iteratively applying a sequence of 2D and 3D deformations to register the stacks into the MRI geometry \cite{Mansoori2007}. This strategy is limited by the complexity of the deformation between MRI and histology images, including 3D non-rigid deformations introduced in the sample preparation process as well as independent 2D deformations for each individual slice.  
  
  Block face images, taken from the surface of the fixated tissue volume before sectioning, provide an intrinsically coherent set of reference images. Their advantage over MRI is that a one-to-one correspondence with histology slices is preserved, uncoupling 2D from 3D deformations. Although suitable for low-resolution alignment  \cite{Palm2010}, the reference images are very different in appearance to the histology and typically have a much lower spatial resolution, hindering accurate alignment. Algorithms have been proposed to smooth out this noise through the more precise and reliable coregistration of adjacent slices, in both a sequential \cite{Yushkevich2006,Chakravarty2008} and simultaneous \cite{Feuerstein2011} manner. Yushkevich et al. attempt to correct for the banana effect in serially registered mouse \cite{Yushkevich2006} and human \cite{Adler2012} brain histological volumes. Their approach involves a combination of rigid-body registrations between neighbouring slices and 3D registration to an MRI scan, with a final smoothing to reduce inter-slice misalignment. Their method reduces the banana effect by increasing the number of inter-slice registrations to include non-adjacent slices. However, it is still limited by the complexity of the combined 2D and 3D deformations, due to the lack of a slice-to-slice correspondence between the volumes, which restricts the choice of geometric transform. Chakravarty et al. \cite{Chakravarty2008} first rigidly align sections of whole mouse brain to an associated block face set, then apply a deformation field transformation to each slice, calculated as the mean of the parameters from deformation fields registering it to the two neighbouring slices. While this reduces noise from the block face registrations, the maximum distance of information transfer is one slice, and so only distortions with high spatial frequencies are corrected.
  
  Techniques have been developed to combine monomodal adjacent slice and multimodal reference registrations simultaneously. Palm et al. \cite{Palm2008} expound a `weighted multi-image mutual information metric', optimising the sum of scaled contributions from both cost functions, and Feuerstein et al. \cite{Feuerstein2011} combine two potential energy functions in a Markov random field model to reconstruct a rat kidney. A limitation of methods that increase the complexity of similarity functions is that multiple local minima are likely to appear, with the risk of making the final result strongly dependent on weighting and initialization.
  
  Finally, \cite{Arganda-Carreras2010} offer a robust approach to combining monomodal and multimodal alignments, using consistent b-spline-based elastic registrations. Multiple iterations of triple-wise registrations gradually share information in the z-direction and result in smoothing through a spectrum of scales. However, the choice of simultaneous registration has the risk of leading to the same problem of multiple local minima mentioned above.
  
  In this paper we introduce ‘transformational diffusion smoothing’ (TDS), a novel technique incorporating reference-based and slice-to-slice registration methods. We apply this method, firstly to a simulated 3D geometry based on a histology slice to which known transformations are applied, and subsequently on a real high-resolution histology volume comprising a whole heart.  Overall, results demonstrate the robust nature of this new approach.
  % \hfill mds
  %  
  % \hfill January 11, 2007
  % 
  
  % needed in second column of first page if using \IEEEpubid
  %\IEEEpubidadjcol

% section introduction (end)
