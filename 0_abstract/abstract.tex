%!TEX root = ../diffusion_paper.tex
\begin{abstract}
Adjacent histological slices can be coregistered accurately and lead to smooth image volumes, owing to their close morphological resemblance and their similar intensity spectra. However, volumes constructed from serial histology registration do not reflect the true 3-dimensional tissue geometry. Registration of histology to an MRI or PET volume or to a set of coherent reference images yields an authentic geometry on the organ scale, yet the lower resolution and differing modality of the references leads to noisy, jagged volumes on the microstructural scale.

We present an algorithm to align neigbouring slices accurately and smoothly without disturbing large scale tissue shape, based on a microscopic model of diffusion. We develop a mathematically sound and general framework of transformational diffusion smoothing (TDS), based on the Lie theory of continous groups. Using synthetic geometries of cardiac tissue with artificial noise, we demonstrate a robust and precise dispersion of information between slices on a configurable range of scales, recovering volumes which are orders of magnitude smoother and which have maintained faithfully the underlying geometrical signal. We apply the algorithm to a rat heart volume that has been reconstructed using block face reference images. We first apply the smoothing globally and then again to the region around an epicardial vessel. Previously indiscernible microvasculature and sheet structure become patent. Pericardium and epicardial vessel segmentations show that displacement aberrations between adjacent slices of the order of 400$\mu$m are reduced by two orders of magnitude. The methods presented here outperform any such method to reconstruct histological volumes based on reference images currently in the literature. Finally, we discuss several interesting applications and refinements that might be made to the algorithm in specific cases, including anisotropic diffusion based on image features or inter-slice metric values or transform magnitudes.
\end{abstract}
